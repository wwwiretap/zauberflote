\documentclass[letterpaper,twocolumn,10pt]{article}

\usepackage{style, verbatim, url, graphicx, xspace}
\usepackage[colorlinks=true, citecolor=blue, linkcolor=blue]{hyperref}
\usepackage{caption}
\usepackage{cleveref, breakurl}

\makeatletter
\newcommand{\crefnames}[3]{
    \@for\next:=#1\do{
        \expandafter\crefname\expandafter{\next}{#2}{#3}
    }
}
\makeatother
\crefnames{part,chapter,section}{\S}{\S\S}

\newcommand{\zbf}{Zauberfl\"{o}te\xspace}
\newcommand{\projtitle}{\zbf: A Transparent Peer-to-Peer CDN}
\newcommand{\inclfigure}[4]{
    \begin{figure}
        \begin{centering}
            \includegraphics[angle=#3,width=.8\linewidth]{#2}
            \caption{#4}
            \label{#1}
        \end{centering}
    \end{figure}
}

\begin{document}

\title{\Large \bf \projtitle}

\author{
    {\rm Anish Athalye}\\
    aathalye@mit.edu
    \and
    {\rm Ankush Gupta}\\
    ankush@mit.edu
    \and
    {\rm Katie Siegel}\\
    ksiegel@mit.edu
}

\maketitle
\thispagestyle{empty}

\begin{abstract}
\zbf is a peer-to-peer system based on the BitTorrent protocol that uses WebRTC
to distribute content between peers. \zbf provides an easily accessible way for
developers to alleviate content delivery bandwidth concerns by marking DOM elements
such that \zbf will fetch these resources from peers with that resource. The \zbf
system has server-side tracker and WebRTC signaling channel components, as well
as client-side scripting that requests and delivers content between peers.
This version of the system implements basic chunking, parallelism, and fault tolerance.
In this paper, we describe the motivation for \zbf and the design and implementation of the
different components that comprise the system. We then detail future improvements
that could be made to increase the robustness and performance of the overall system.
\end{abstract}

\section{Introduction}
It is common for personal blogs and other small websites to gain sudden attention
on large, high-traffic websites such as reddit. Since these websites usually
experience much smaller traffic loads, the hardware devoted to hosting them
is typically not powerful enough to cope with this large influx of content requests.
These sudden, unexpected bursts of traffic often result in websites loading at
extremely slow speeds, or not loading at all, and is colloquially known as the
``reddit hug of death'' or the ``reddit effect.'' \\

Typical means for dealing with this situation involve utilizing Content Delivery
Networks (CDNs), but these are usually out of the budgetary constraints for
websites which only experience such loads on an occasional basis. An ideal
solution would be a lightweight means to mimic the benefits of an ordinary
CDN, which is typically backed by a robust network of machines. \\

In this paper, we present \zbf, a peer-to-peer CDN. \zbf provides an easy way
for developers to distribute content to clients via a peer-to-peer network,
rather than directly from the central server that hosts the content. \zbf
is comprised of a server-side component and a client-side component. The
centralized, server-side component both tracks resource propagation metadata
and acts as an assist for opening peer-to-peer data channel connections. The
lightweight client-side component requests resources from peers and distributes
resources in response to peer requests. \\

\zbf's design is based on the BitTorrent protocol~\cite{cohen:bittorrent}, by which
clients requesting content attempt to receive chunks of data from other
clients, rather than retrieving the entire file from a central server. As
with BitTorrent trackers, any developer could host their own tracker and service
it to connect peers requesting a resource with peers distributing the resource. \\

The \zbf client-side code runs purely in the browser, and is completely transparent
to the user. The user does not have to install any application on their
computer to obtain desired resources peer-to-peer. As a result, the \zbf client
code only runs when a user is currently visiting the website, therefore adding
no burden to the user from constant background processes. \\

In designing \zbf, we took into consideration the unique problems presented
by browser-to-browser content streaming. Website visitor churn is often very
high with any type of website; our system had to be robust under these common
conditions. \zbf forms peer-to-peer connections quickly and immediately streams
requested resources through these data channel connections. \zbf also recovers
quickly from data request failures; the \zbf data chunking component minimizes
latency from these failures and quickly retries resource requests with other
peers.

\section{\zbf API}

\zbf supports peer-to-peer loading for CSS (via \texttt{link} elements),
Javascript (via \texttt{script} elements), and images (via \texttt{img} elements).
For an existing website to utilize \zbf, the developer simply needs
to make two changes to the DOM elements. \\

The first change is renaming the \texttt{src} or \texttt{href} attribute to
\texttt{data-zf-fallback}. \zbf uses the \texttt{data-zf-fallback} attribute
in order to (1) prevent the browser from automatically fetching the resources,
and (2) provide a URL to be used in the event that the peer-to-peer network
is unable to serve the content. \\

The second attribute that must be present on a DOM element that is fetched by
\zbf is the \texttt{data-zf-hash} attribute. The \texttt{data-zf-hash} attribute
is the SHA1 hash of that resource, which can be computed by the developer, and
is utilized by \zbf to ensure integrity of data fetched via peer-to-peer interaction.

\section{Implementation}

The \zbf architecture consists of several loosely-connected components (see
\Cref{fig:components}). At the highest level, there is a download manager that
provides a reliable download service. It relies on the tracker interface and
the connection manager. The tracker interface facilitates communication with
the central tracker for information about peers and file sizes, and the
connection manager provides an interface for peer-to-peer message transfer. \\

Our components rely on two standard APIs provided by modern browsers,
WebSockets~\cite{w3c:websocket} for client-server communication and
WebRTC~\cite{w3c:webrtc} for peer-to-peer communication.

\inclfigure{fig:components}{components.pdf}{0}{
    Relationship between different client-side components of the \zbf
    peer-to-peer download service. Components in gray are standard browser
    APIs.
}

\subsection{Connection manager}

The connection manager is responsible for establishing connections between
peers and facilitating peer-to-peer message transfers. It is a thin layer on
top of WebRTC that handles maintaining connections with multiple peers,
associating peers with peer IDs that are assigned by our tracker. It manages
all signaling through the signaling channel, which implements signaling through
our central server using WebSockets. At this time, this signaling server is the
same server that is running the tracker, but they are logically separate
components, so theoretically, they could be decoupled if the need arises for better load
balancing.

\subsection{Download manager}

The download manager is the top-level client interface for peer-to-peer
downloads. It provides an API by which the client can provide a hash, and
optionally, a fallback HTTP URL. The download manager will asynchronously
download the content associated with the hash and call a user-provided callback
function when finished. The download manager implements reliable download,
preferring peer-to-peer over HTTP when possible.

\subsubsection{Chunking}

\zbf divides assets into chunks; chunks are individually requested and
sent between peers. We use a standard chunk size of 10240 bytes in our
implementation; preliminary tests showed this chunk size to be reasonable.
When a client asks for a resource, the download manager for that client
determines the number of chunks for that resource from the resource's total size.
The download manager then evenly distributes requests for these chunks
among the peers with the resource. \\
Requests for chunks time out on a per-chunk basis. We track the time \zbf
last sent a request for each chunk; if this time exceeds a cutoff, we randomly
choose another peer with the resource and send the chunk request to that
peer. We periodically retrieve an updated peers list for this resource
from the tracker. \zbf uses exponential backoff when sending chunk requests
so we don't flood peers with requests as a result of slow network connectivity.

\subsubsection{Hash validation}

When fetching content peer-to-peer, it is necessary to authenticate data that
is downloaded. \zbf achieves this by using a cryptographic hash function to
authenticate data that is downloaded. Assets that are to be downloaded
peer-to-peer are identified by their hash, and once the download manager
finishes downloading the content from peers, it verifies that the hash of the
contents matches what is expected. The download manager rejects content if
there is an authentication error. In the current implementation, if this
occurs, the download manager falls back to downloading the content over HTTP
from the fallback URL.

\subsubsection{Parallelization}

\zbf divides assets into chunks primarily to enable parallelization. Because
individual peers may not have high outbound bandwidth, it is beneficial to
download in parallel from many peers. Our implementation parallelizes chunk
downloads, with pending chunk requests balanced between available peers in the
network. If certain peers are slow to respond, the download manager
automatically requests chunks from faster peers to optimize download time.
Parallel downloads help reduce download latency.

\subsubsection{Fault tolerance}

Our download manager implements reliable download semantics. The implementation
is resilient to peer failure, automatically re-requesting chunks from live
peers when necessary. In the worst case, even if all peers go down, the
implementation falls back to downloading over pure HTTP from the centralized
server as a last resort. In any case, the download manager guarantees that the
data will be downloaded and that the download will be authenticated.

\subsection{Tracker interface}

Via the tracker interface, the download manager can request a list of peers that
have a certain resource, specifying the desired resource via that resource's
unique SHA1 hash. After a client receives the full resource, the download
manager publicizes that this client has the resource via this tracker interface,
letting the central tracker know that it possesses the resource.

\subsection{Signaling channel}

The signaling channel facilitates opening the WebRTC data channel connection between
peers. The WebRTC protocol requires one peer to send a WebRTC offer and ICE candidate
to the other peer, and that other peer to send back a WebRTC answer, for a peer-to-peer
data channel to be opened. The server-side signaling channel facilitates this
information transfer between peers by directing offers, ICE candidates, and answers
to their specified peer recipient.

\subsection{DOM injection layer}
The DOM injection layer facilitates the link between the DOM as displayed in the browser and the entire peer-to-peer system. The DOM injection layer creates a WebSocket, and then instantiates a Tracker and Signaling Channel. It uses these to then instantiate a Connection Manager and in turn, uses this to instantiate a Download Manager.

The DOM injection layer then iterates through all the DOM elements with the \texttt{data-zf-hash} and \texttt{data-zf-fallback} attributes. It the values of these attributes to fetch the data via the Download Manager. Upon receiving the data for each web element, the DOM injection layer creates an in-memory Blob and generates URL for this object. It then obeys the following rules, which can easily be expanded to support other DOM elements:

\begin{description}
\item[Script elements] have the \texttt{type} of their Blob set to \texttt{text/javascript}. They then have their \texttt{src} attribute set to the URL of the in-memory Blob.
\item[Link elements] have the \texttt{type} of their Blob set to \texttt{text/css}. They then have their \texttt{href} attribute set to the URL of the in-memory Blob, and \texttt{rel} attribute set to \texttt{stylesheet}.
\item[All other elements] have the \texttt{type} of their Blob set to \texttt{application/octet-stream}. They then have their \texttt{src} attribute set to the URL of the in-memory Blob. As most browsers can intelligently determine the content of \texttt{application/octet-stream} data, this functions properly for Image nodes, and serves as the generic template for any other node.
\end{description}

\subsection{Server implementation}

There are two main server-side components to \zbf. First is the tracker,
which maintains a mapping from resource hashes to the peers that have that
resource. Second is the WebRTC handshake channel, which passes information
between peers to enable them to open WebRTC data channels.

\subsubsection{Tracker}
The tracker is hosted on a remote server and tracks client connections.
The client-side tracker interface opens a WebSocket connection with the remote tracker,
which registers the existence of this client. The remote tracker responds to
requests from the client for lists of peers with a given resource. When a client
publicizes that it has a resource through the tracker interface, the remote tracker updates
the list of peers with the resource to include this client, as
long as its WebSocket connection remains open. Upon WebSocket disconnection,
the tracker ``forgets'' about the client and erases all data about the client's
held resources.

\subsubsection{WebRTC handshake channel}

WebRTC handshake information--the offers, ICE candidates, and answers--are
sent through the client-side signaling channel through a websocket to
the server. The client-side signaling channel specifies a peer to which
this WebRTC handshake information should be sent; if the server has an open
websocket connection with that peer, the server passes that message to the peer.

\section{Evaluation}

We tested \zbf on a simple site that loads one large 768KB image from a remote server.
In our tests, we compare page load times using both plain HTTP and \zbf in a range of
scenarios. Specifically, we varied two parameters: the number of seeding peers and the
number of leeching peers. A seeding peer is a peer that has the desired resource, which
in our tests, was the single image on the site. A leeching peer is a peer that desires the
resource, so must request it either from the central site host or from peers with the resource.

Our tests launched a certain number of seeding peers, then launched a number of
leeching peers, measuring the average load latency for the leeching peers. We compare
this latency data with the page load latency when the leeching peers must load the content from the
central site host.

\subsection{Simulating the ``reddit effect''}
Our tests involved programmatically opening browser windows and measuring load times.
Due to limitations of the number of browser windows that can be reasonably opened
and the memory overhead of Chrome, our browser testing environment of choice, we could
not reasonably open enough browser windows to exactly imitate millions of simultaneous
visitors to a site. Instead, we simulated the ``reddit effect'' by throttling the content delivery
bandwidth of the central server hosting the desired resource. We limited the content delivery
rate to 64 KB/s and performed benchmark tests that compared the performance of \zbf
against the HTTP content delivery latency under the throttled bandwidth conditions.


\subsection{Testing infrastructure}

% TODO selenium with chromedriver & scala
% anish should probably do this section

\subsection{Results}

\subsubsection{Scalability of \zbf}
The first empirical analysis that we performed was determining how page load time
over \zbf scales as we vary the number of users who have the page loaded (we call
these users \textit{seeders}) and the number of users attempting to concurrently
download the page (we call these users \textit{leechers}).

\inclfigure{fig:heatmap}{heatmap.eps}{-90}{
    Average page load time over P2P for seeders vs leechers.
}

As seen in \Cref{fig:heatmap}, the performance of \zbf depended on the
seeder-to-leecher ratio. In general, the lowest latency in load time was
achieved when the seeder-to-leecher ratio was highest. However, the system
performed well in any condition with many seeder peers, reflecting the
performance benefits provided by chunking of resources and parallelization of
sending chunks. The system performed worst when there were many leechers and
only a few seeders.



\subsubsection{\zbf vs HTTP}

Our next empirical analysis was to keep the ratio of seeders to leechers
constant at 1, and compare the page load times for a throttled HTTP server vs
\zbf as we increase the number of users attempting to concurrently load the
page.

\inclfigure{fig:scalability}{scalability.eps}{-90}{
    Average page load time over P2P compared to HTTP on throttled server
    assuming constant ratio of 1 seeder per leecher.
}

As seen in \Cref{fig:scalability}, \zbf performed well in comparison to
loading content over HTTP. While both experienced increases in content delivery
latency with increasing numbers of peers requesting content, in general \zbf
had lower latency than HTTP. Note that our simulated ``reddit effect'' implies
hundreds of thousands of other simultaneous connection; however, since we are
simulating the effect, these imaginary peers cannot distribute content to new
peers. So, \zbf could potentially provide an even greater latency drop with
more active seeding peers, which would be the case in a real-world scenario.

\section{Future work}

There remains numerous optimizations that would make \zbf more performant
when used on a deployed, high-traffic site. We detail these potential optimizations
in this section, along with features which could allow \zbf to handle more use cases.

\subsection{Per-chunk hash validation}
As of right now, \zbf verifies the integrity of the downloaded resource once all
chunks have been retrieved and joined. In normal circumstances where all chunks
successfully download, this may not prove to be an issue. However, waiting until
all chunks are downloaded and combined to verify the file means that if a single
chunk had been corrupted or compromised, all chunks for the file would need to be
downloaded, combined, and verified again. \\

An improvement to \zbf that would mitigate this issue would be to have the Tracker
module provide checksums for each chunk. From the user's perspective, the
\texttt{data-zf-hash} attribute currently used to store the hash of the file
would be replaced with the hash of the JSON object storing a list of the hashes
for each chunk. Upon connection to the tracker, each client would receive that
list and verify its integrity, and then be able to verify and redownload each chunk
independently in the event of failure.

\subsection{Hybrid downloads over HTTP and P2P}

Another improvement to \zbf would be to allow it to parallelize downloads via HTTP and P2P transfers simultaneously. By using the \texttt{Range} HTTP header attribute, \zbf could fetch the same data as contained in P2P chunks from the HTTP fallback server. By utilizing both HTTP and P2P, \zbf may be able to provide service speed that is expected in all instances to be faster than simply using HTTP alone.

\subsection{Variable request sizes}

Implementing variation in requested chunk sizes could allow \zbf to independently
determine a more optimal chunk size for each asset. For instance, we hypothesize
that different chunk sizes are optimal for differently sized assets, and potentially
for different media types. If \zbf more intelligently determined the optimal size
of chunks for each resource, overall download latency would decrease.

\subsection{Bandwidth detection and prioritization}

If \zbf could detect which peers have higher bandwidth connections, and thus
are more capable of swiftly delivering data to a client, \zbf could prioritize
requesting data from these peers. Perhaps \zbf could request larger chunks from
peers with higher bandwidth connections. \zbf could also dynamically react to
poor connections with peers during the chunk requesting process; chunk requests
that subsequently time out could be redistributed more intelligently among
peers with higher bandwidth connections.

\subsection{Support for HTML5 media}
It is possible to add functionality such that \zbf can construct
\texttt{ReadableStream}s (as specified by the Streams standard)
%TODO: cite https://streams.spec.whatwg.org
from the chunks of data as it receives them. Using this stream, the DOM injection
layer could feasibly replace HTML5 media elements such as embedded Video and Audio.
Since these types of media are particularly large in size, being able to distribute
them in a decentralized, peer-to-peer manner could provide substantial load
reduction for the central server.

\subsection{Automatic conversion to \zbf}
Developers can integrate their sites to use \zbf by adding attribute tags to
the site resources they wish to be distributed using the system. However, sites feature
dozens of resources; as a result, adapting a site to use \zbf could present a
burden on the developer. If we were to provide a script that developers could use
to change HTML such that all resources were loaded using \zbf, this would ease
integration pains and increase adoption.

\section{Related work}

%! TODO cite peercdn in bibtex
Currently, alternative peer-to-peer content delivery networks do exist. The three
prominent peer-to-peer CDNs are PeerJS, PeerCDN, and Peer5. PeerJS is open-sourced
and similarly transparent, but only encapsulates the functionality of our connection manager
module. PeerCDN~\cite{peercdn} is a closed-source alternative providing  similar
functionality to \zbf; it offloads site content delivery to users of a site,
significantly lowering strain on the central source. However, PeerCDN was acquired by
Yahoo! and no longer exists. Finally, Peer5 is a service used by companies today.
Unfortunately, Peer5 is a paid service that is entirely closed-source; the Peer5 code is
not transparent to the clients on which it runs.

\section{Conclusions}

\zbf is an open-sourced system that will allow developers to easily utilize
a peer-to-peer CDN to increase the content delivery bandwidth of their sites,
even when operating under heavy request loads. Preliminary data shows that
the system performs well under medium and high workloads, and when the
seeder-to-leecher ratio is high. Future work could refine the system
such that it works well in a wide variety of scenarios. Our results show that peer-to-peer
CDNs are a promising low-cost alternative to CDNs that require on an extensive
network of server and server infrastructure.



{\footnotesize \bibliographystyle{ieeetr}
\bibliography{paper}}

\end{document}
