\documentclass[letterpaper,twocolumn,10pt]{article}

\usepackage{style, verbatim, url}
\usepackage[colorlinks=true, citecolor=blue]{hyperref}
\usepackage{cleveref}
\usepackage{breakurl}

\makeatletter
\newcommand{\crefnames}[3]{
    \@for\next:=#1\do{
        \expandafter\crefname\expandafter{\next}{#2}{#3}
    }
}
\makeatother
\crefnames{part,chapter,section}{\S}{\S\S}


\begin{document}

\title{\Large \bf Zauberfl\"{o}te: A Transparent Peer-to-Peer CDN}

\author{
    {\rm Anish Athalye}\\
    aathalye@mit.edu
    \and
    {\rm Ankush Gupta}\\
    ankush@mit.edu
    \and
    {\rm Katie Siegel}\\
    ksiegel@mit.edu
}

\maketitle

\thispagestyle{empty}

\section{Introduction}


Currently, peer-to-peer content delivery networks exist, yet are usually
closed-source. PeerCDN is one such example found in industry; it offloads site
hosting to users of a site, significantly lowering strain on content delivery
from a central source. In our project, we aim to create an open-sourced system
that will allow developers to easily integrate a peer-to-peer CDN to expedite
content delivery to clients even when operating under heavy load. We plan to
base our design on the BitTorrent protocol~\cite{cohen:bittorrent}, where
clients requesting content will attempt to receive chunks of data from other
clients rather than a central server. We plan to begin with tracker software
and then add client software for communicating with the tracker and sharing
content with peers.

\section{Design}

We plan to implement a peer-to-peer CDN using a hybrid HTTP/P2P setup. For the
peer-to-peer part of the service, we will use a BitTorrent-like protocol to
share data between clients. We will use intelligent client software that uses
both HTTP and P2P to minimize latency and maximize throughput.

We will have a centralized tracker, a server that keeps track of file metadata
and facilitates matchmaking between peers. This server will not host the file
contents themselves. The server will make use of cryptographic hash functions
to guarantee the integrity of data and prevent malicious clients from
distributing bad data.

We will have a separate HTTP server to serve content and bootstrap peers when
there are no other clients with a copy of the content.

On the client side, we plan to use only HTML5 APIs to communicate with our CDN.
This will make it simple to start using the CDN, and all clients with
up-to-date browsers will just work with the service. The process will be
entirely transparent to end users, because it will not require the use of
browser extensions or other custom software that has to be manually installed.
We plan to make use of WebRTC for peer-to-peer functionality~\cite{w3c:webrtc}.

\section{Goals}

Zauberfl\"{o}te provides basic framework of peer-to-peer content delivery on
which other systems can be built. We plan to build in the initial
infrastructure and then conduct tests to evaluate the performance of the
system.

To demonstrate the robustness of our peer-to-peer CDN system, we plan to
measure the latency of several large data transfers and compare this
performance to that of a simple centralized system. Our results will help
identify scenarios in which such peer-to-peer CDN systems enhance the
performance of an application.

After measuring the performance of our system, we intend to work on decreasing
the latency involved with obtaining large resources on the internet by allowing
data transfer from peers, rather than a central server, when operating under
heavy load. At the same time, we want to attempt to ensure that page loads are
not any slower than they would be with the use of only a central server by
heuristically evaluating the load times of data via peer-to-peer transfers as
compared to downloading from the central server.

Potential extensions include peer-to-peer video streaming and multiplayer
gaming. We hope that the latency decrease in such applications would be
significant, leading to a better overall user experience.

{\footnotesize \bibliographystyle{ieeetr}
\bibliography{paper}}

\end{document}
